A library to convert Roman numerals to their integer values.

\subsection*{Building the library }

To build the static library, you\textquotesingle{}ll need the following commands available in your system\+:


\begin{DoxyItemize}
\item {\ttfamily g++} (C++ compiler)
\item {\ttfamily ar} (static library creator)
\item {\ttfamily pthreads}
\item {\ttfamily cmake}
\end{DoxyItemize}

In addition, to run the tests you need to have the G\+Test framework in your system. The easiest way is to clone the \href{https://github.com/google/googletest}{\tt Google Test repository} to a directory in your local path and then build it using {\ttfamily cmake}\+:


\begin{DoxyCode}
1 cd /home/foo/
2 git clone https://github.com/google/googletest
3 cd googletest
4 mkdir build && cd build
5 cmake .. && make
\end{DoxyCode}


You can then {\ttfamily cd} to this project root folder {\ttfamily libroman} and pass the G\+Test root directory as the variable {\ttfamily G\+T\+E\+S\+T\+\_\+\+R\+O\+O\+T\+\_\+\+D\+IR} to the {\ttfamily make} command. 
\begin{DoxyCode}
1 cd /home/foo/libroman
2 make GTEST\_ROOT\_DIR=/home/foo/googletest
\end{DoxyCode}


If not, you\textquotesingle{}ll need to provide the paths to G\+Test as variables to the {\ttfamily make} command\+:


\begin{DoxyCode}
1 cd /home/foo/libroman
2 make GTEST\_LIB\_DIR=/home/foo/gtest-lib-path GTEST\_INCLUDE\_DIR=/home/foo/gtest-include-path
\end{DoxyCode}



\begin{DoxyItemize}
\item {\ttfamily G\+T\+E\+S\+T\+\_\+\+L\+I\+B\+\_\+\+D\+IR} points to where the library .a file is. Defaults to {\ttfamily G\+T\+E\+S\+T\+\_\+\+R\+O\+O\+T\+\_\+\+D\+I\+R/build/googlemock/gtest}.
\item {\ttfamily G\+T\+E\+S\+T\+\_\+\+I\+N\+C\+L\+U\+D\+E\+\_\+\+D\+IR} must point to where the include directory {\ttfamily gtest} is. {\ttfamily Defaults to G\+T\+E\+S\+T\+\_\+\+R\+O\+O\+T\+\_\+\+D\+I\+R/googletest/include}.
\end{DoxyItemize}

To run the tests, simply run {\ttfamily make run-\/tests}.

\subsection*{Using the library }

To use the library, simply include the header \hyperlink{roman_8h}{roman.\+h} and call the function \hyperlink{roman_8c_a5d15ad3ed29e4dc0fed9b718523c48c8}{roman\+\_\+to\+\_\+int()} with a string as its only parameter\+:


\begin{DoxyCode}
\textcolor{preprocessor}{#include <stdio.h>}
\textcolor{preprocessor}{#include "\hyperlink{roman_8h}{roman.h}"}

\textcolor{keywordtype}{int} \hyperlink{roman-to-int_8c_a0ddf1224851353fc92bfbff6f499fa97}{main}() \{
    \textcolor{keywordtype}{int} value = \hyperlink{roman_8c_a5d15ad3ed29e4dc0fed9b718523c48c8}{roman\_to\_int}(\textcolor{stringliteral}{"XII"});
    printf(\textcolor{stringliteral}{"%d\(\backslash\)n"}, value); \textcolor{comment}{// 12}
\}
\end{DoxyCode}


And then you need to compile your program linking it against the {\ttfamily libroman.\+a} static library in the {\ttfamily lib/} directory, including the header directory as well\+:


\begin{DoxyCode}
1 g++ -o bar bar.c -I /home/foo/libroman/include -L /home/foo/libroman/lib -lroman
\end{DoxyCode}


\subsection*{Using the standalone binary }

There\textquotesingle{}s a standalone binary {\ttfamily roman-\/to-\/int} in the {\ttfamily bin/} directory that accepts Roman numerical strings as its only command-\/line argument, and prints to stdout its numerical integer value. For example, the code below outputs {\ttfamily 12}\+:


\begin{DoxyCode}
1 cd /home/foo/libroman
2 ./bin/roman-to-int XII
\end{DoxyCode}


\subsection*{Documentation }

To read the docs, go to {\ttfamily doc/html/index.\+html} in your browser or open the file {\ttfamily doc/latex/refman.\+pdf}.

\subsection*{Roman numerical system }

The Roman numerical system is composed by assigning numerical values to certain characters, as in the table below\+:

\tabulinesep=1mm
\begin{longtabu} spread 0pt [c]{*2{|X[-1]}|}
\hline
\rowcolor{\tableheadbgcolor}{\bf Character }&{\bf Value  }\\\cline{1-2}
\endfirsthead
\hline
\endfoot
\hline
\rowcolor{\tableheadbgcolor}{\bf Character }&{\bf Value  }\\\cline{1-2}
\endhead
I &1 \\\cline{1-2}
V &5 \\\cline{1-2}
X &10 \\\cline{1-2}
L &50 \\\cline{1-2}
C &100 \\\cline{1-2}
D &500 \\\cline{1-2}
M &1000 \\\cline{1-2}
\end{longtabu}
The numbers must normally be written in descending order, with repeating characters having their value multiplied by the number of times they appear (3, at most). As such, we\textquotesingle{}d have VI = 6, M\+M\+M\+LX = 3060, for instance. The characters V, L and D cannot be repeated in any circunstance, rendering Roman numbers such as VV or M\+D\+DI invalid.

If a character like I, X or C appears before another with a greater value, its value is subtracted from this one\+: IV = 4, for instance. The rules for subtracting characters are written in the table below\+:

\tabulinesep=1mm
\begin{longtabu} spread 0pt [c]{*2{|X[-1]}|}
\hline
\rowcolor{\tableheadbgcolor}{\bf Character }&{\bf Can appear before  }\\\cline{1-2}
\endfirsthead
\hline
\endfoot
\hline
\rowcolor{\tableheadbgcolor}{\bf Character }&{\bf Can appear before  }\\\cline{1-2}
\endhead
I &V, X \\\cline{1-2}
X &L, C \\\cline{1-2}
C &D, M \\\cline{1-2}
V, L, D &None \\\cline{1-2}
\end{longtabu}
When subtracting there can\textquotesingle{}t be any repetition, thus I\+IX or C\+C\+MI are invalid Roman numbers. 